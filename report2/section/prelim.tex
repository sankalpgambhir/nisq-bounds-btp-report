% prelim.tex

% classical prelims
\subsection{Classical Computing}
% learning

% how optimise
\subsubsection{Optimisation Techniques}
The discussion of optimization techniques in classical computing is a long and
arduous one. We refer the reader to a common text on the matter for a detailed
discussion \cite{boyd2004convex,nocedal2006numerical}, while reviewing the
general ideas briefly here.

% actual discussion of common techniques
The goal of an optimization problem, given (generally) a function
\(\loss:X\to\reals\) is to find the minimum value in its range, and sometimes an
inverse in the domain, i.e., output a point \(x_* \in X\), such that

\begin{gather*}
    x_* = \text{arg min } \loss(x)~,
\end{gather*}

or the argument to \(\loss\) which minimizes it. In some problems, a local
minima may suffice, and in others, a global minimum may be the requirement.
Depending on the nature of the domain \(X\), different techniques may be
employed to find \(x_*\). These include gradient descent and its reductions
(finite element methods, etc.), Hessian-aided descent,  stochastic gradient
descent, among others. The rest of the discussion is agnostic of the
optimization routine, beyond, of course, the existence of one.

In a NISQ system, there is generally little to no attempt at error correction,
and the general goal is to capitalize on what is possible with the short
available coherence times, without devoting a majority of the system's resources
to error checking and correction. As such, these systems are unable to support
high-depth circuits with computationally involved analytical gradient based
approaches. An effective optimizer in control of such a temporally-bound circuit
should try to utilize techniques minimizing the number of measurements or
function evaluations, as the relevant modules generally form the bottleneck of
the computation \cite[see][chapter II.D]{bharti2021noisy}.

% qm prelims
\subsection{Quantum Regime}

% hilbert space
\subsubsection{Hilbert Space}
\begin{definition}
    A Hilbert space is a vector space \(\hilbertspace\) equipped with an inner
    product \(\innerproductabstract{f}{g} \;\forall f, g \in \hilbertspace\) such that
    the norm defined by

    \begin{equation*}
        \norm{f} = \sqrt{\innerproductabstract{f}{f}}
    \end{equation*}

    turns \(\hilbertspace\) into a complete metric space
    \cite{sansone1959orthogonal}.
\end{definition}

Physical quantities --- such as energy, momentum, and position --- are
represented as operators over a Hilbert space \(\hilbertspace\) to which the
wavefunctions belong \cite{hall2013quantum}. 
For our purposes, we will assume \(\hilbertspace\) to always
have as its base field the field of complex numbers, \(\complex\).

Herein, the complex inner product is assumed to be linear in the second factor,
i.e.,

\begin{equation*}
    \innerproductabstract{f}{\lambda g} = \lambda \innerproductabstract{f}{g}; \quad \innerproductabstract{\lambda f}{g} = \conjugate{\lambda} \innerproductabstract{f}{g}
\end{equation*}

\(\forall f, g \in \hilbertspace\) and \(\lambda \in \complex\). 

For every bounded operator \(A\) acting on a Hilbert Space, there is a unique
bounded operator \(A^*\) called its \emph{adjoint} such that

\begin{equation*}
    \innerproductabstract{f}{Ag} = \innerproductabstract{A^*f}{g}~.
\end{equation*}

We will assume relevant quantities to be linear operators \(\cdot :
\hilbertspace \to \hilbertspace\) with adjoints where necessary,
\cite[see][Appendix A]{hall2013quantum} for details.

% schrodinger
% \subsubsection{Schr\"odinger Equation}
% Given a \emph{Hamiltonian} operator \(\hamiltonian\) for a system represented by
% a wavefunction \(\psi\), its time evolution is given by the Schr\"odinger
% equation,

% \begin{equation}
%     \frac{\partial{\psi}}{\partial{t}} = \frac{1}{\iota\hbar} \hamiltonian \psi~.
%     \label{eq:schrodinger}
% \end{equation}

% quantum computer
\subsubsection{Quantum Computation}
In a quantum system, `computation' in its essence is jugglery of probability
amplitudes of states using unitary actions. After action of unitaries as
necessary, the coefficients are estimated using a measurement schema and
post-processed to recover the computational result. See
\cite{nielsen2002quantum} for a detailed study.

% quantum embedding
\subsubsection{State Construction and Embedding}
To perform computation in the quantum regime, data first needs to be converted
to a format in which it can be acted upon by quantum operators. This is again an
embedding function, specifically from the input domain to the Hilbert space of
quantum states for the system ansatz. See \cite{lloyd2020quantum} for details.

% distance measures
\subsubsection{Information Matrices and Distance Measures}
\label{subsubsec:distanceinfo}

Consider a parametrized distance function on quantum states 

\begin{equation}
    d(\parameters, \parameters') = d(\rho(\parameters), \rho(\parameters'))~.
\end{equation}

For close enough \(\parameters\) and \(\parameters'\) we can write a Taylor
expansion of the distance function. Clearly, the zeroth and first terms vanish,
as the distance of a state from itself is zero, and it forms a minimum for the
distance function. We have then the second order term

\begin{equation}
    d(\rho(\parameters), \rho(\parameters+\perturbdel)) = \frac{1}{2} \perturbdel^\top F(\parameters)\perturbdel~,
\end{equation}

with \(F\) representing the metric tensor

\begin{equation}
    F(\parameters) = \frac{\partial^2}{\partial\delta_i\partial\delta_j} d(\parameters, \parameters + \perturbdel) \bigg\vert_{\perturbdel = 0}~.
\end{equation}

Therefore, \(F\) captures all the information necessary to define a distance
metric. \(F\) is called the Quantum Fisher Information Matrix
\cite{meyer2021fisher}, and it is defined as

\begin{equation}
    \left[F_{ij}\right] = 4\text{Re} \left[\braket*{\partial_i\psi(\parameters)}{\partial_j\psi(\parameters)} - \braket*{\partial_i\psi(\parameters)}{\psi(\parameters)}\braket*{\psi(\parameters)}{\partial_j\psi(\parameters)}\right]~.
\end{equation}
