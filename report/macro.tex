% general formatting and stuff
\usepackage{graphicx}
\usepackage{xcolor}
\usepackage{ragged2e}
\usepackage{ifthen}
\usepackage{lineno}
\usepackage{booktabs}
\usepackage{subcaption}
\usepackage{multicol}
\usepackage{lipsum} % just to check formatting
\usepackage{titletoc}
\titlecontents{subsubsection}
              [3.3em]
              {\footnotesize\protect\addvspace{5pt}}%
              {\thecontentslabel~~}
              {}%
              {\hfill\contentspage}%

% subject specific
\usepackage{amsmath, amssymb, amsthm}
\usepackage{physics, physconst, physunits}
\newtheorem{axiom}{Axiom}
\newtheorem{definition}{Definition}

% drawing
\usepackage{pstricks}
\usepackage{tikz, tkz-orm, pgfplots}
\usetikzlibrary{arrows,shapes,backgrounds,decorations.markings}
\usetikzlibrary{matrix,positioning,decorations.pathreplacing,calc,tikzmark}
\pgfplotsset{width=7.5cm,compat=1.16}
\usepgfplotslibrary{fillbetween}

% links and citations
\usepackage{hyperref, bookmark}
\hypersetup{hidelinks} % hide boxes around links ew
\usepackage{csquotes}
\usepackage[
    style=numeric,
    sorting=none
]{biblatex}
\addbibresource{biblio.bib}

% notes
\newcommand{\notes}[1]{}

\ifthenelse{\value{notes} > 0}{
    \renewcommand{\notes}[1]{{\textcolor{red}{[Note: #1]}}}
}{}

% actual macros
\newcommand{\reals}{\ensuremath{\mathbb{R}}}
\newcommand{\naturals}{\ensuremath{\mathbb{N}}}
\newcommand{\complex}{\ensuremath{\mathbb{C}}}
\newcommand{\featurespace}{X}
\newcommand{\labelset}{L}
\newcommand{\innerproductabstract}[2]{\langle #1, #2\rangle}
\newcommand{\conjugate}[1]{\bar{#1}}
    % svm
\newcommand{\vecw}{\vec{w}}
\newcommand{\vecx}{\vec{x}}
\newcommand{\vecb}{\vec{b}}
\newcommand{\innerprod}[2]{\langle #1 \dotproduct #2 \rangle}
\newcommand{\trainset}{\mathcal{S}}
    % quantum
\newcommand{\hilbertspace}{\ensuremath{\mathcal{H}}}
\newcommand{\hamiltonian}{\ensuremath{\hat{H}}}
\newcommand{\parameters}{\vec{\theta}}
\newcommand{\perturbdel}{\vec{\delta}}
\newcommand{\loss}{\mathcal{L}}
\newcommand{\pqc}{U}
\newcommand{\genset}{\mathcal{G}}
\newcommand{\dla}{\mathfrak{g}}
