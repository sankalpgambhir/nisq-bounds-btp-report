% intro.tex

% intro to problems of learning and classification
There has been long standing interest in constructing systems capable of
learning from experience since even before computers in their modern form have
existed. In the last few decades, with computing power skyrocketing
exponentially coupled with leaping advances in theory of learning systems and
statistical inference, these problems became tractable and eventually came into
use ubiquitously. With applications ranging from facial detection systems for
surveillence to identifying cosmic objects for cosmology, they have found
widespread adoption in industry and academia. These systems circumvent the need
to produce a precise mathematical model for the problem at hand, exploiting
general techniques to instead infer a model from available data. With their
advent, however, has come an ever rising need for computing power to facilitate
their operation. This has found data centers of unprecendented scales consuming
enormous amounts of power to provide the instant predictions we've come to rely
on.

% intro to how quantum may solve it
With snowballing energy and space requirements of classical computers in the
form of GPU clusters and Application Specific Integrated-Circuits (ASICs), there
has been a spark of interest in offloading this computation onto quantum
computers, which, till recently, have largely remained a rare species spotted
only in labs surrounded by helium-cooled superconductors and white-coated
predators. Current scales of available quantum computers, however, still lack
the power required to fully tackle these challenges while maintaining reliable
error-levels or adding their own error checking and correction. This has
motivated using quantum computers to run bottnecked computational subroutines
with classical control systems. These systems generally lack error correction,
and thus earn themselves the title of `noisy'. These form the basis of
computation considered in this thesis, Noisy Intermediate-Scale Quantum (NISQ)
computers.

Connecting a quantum computer to a classical puppeteer is not expected to come
without its own issues either. It constrains the architecture and is itself
bottlenecked on both ends, first by the parameter transfer and configuration
from the classical to the quantum, then finally by the detectors on the quantum
side to the classical. In this thesis, we focus on the former, discussing the
limits of computation and computational precision achievable with this hybrid
architecture.

% report structure
\subsection{Structure}
In \autoref{sec:prelim}, definitions and relevant results in classical
computing, physics, and quantum information are presented. \notes{Extend this.}

\subsection{Outline of New Results}
\notes{Add summary of results at the end.}
