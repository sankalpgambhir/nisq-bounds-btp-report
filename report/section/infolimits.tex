% infolimits.tex

Having proposed a structure to solve learning problems with a NISQ system, we
are ill-fated to then deal with the myriad of issues that arise with daring to
use it. Barring issues specific to the quantum states, low coherence times,
inaccurate measurements, et cetera \notes{cite something}, our structure itself
imposes some bottlenecks on what we can achieve with it. In particular, the act
of converting and tranferring data between the representations used by the
quantum and classical halves induces the issue of information transfer and
limits upon it. 

The idea of information theoretic bounds goes back to the father of information
theory, Claude Shannon himself \notes{cite shannon}. Work from Nyquist, Hartley,
and Shannon \notes{cite} built up the structure of information theory to
quantify the maximum `amount of data' that could be tranferred through a noisy
communications channel and the changes to said effective `amount' on the
implementation of error correcting codes over the channel.

The analysis for the hybrid computing case has partially been performed in a
general setting \cite{lloyd2014information}. The authors suggest in the paper an
extension to construction of unitaries, but do not explore it further. To the
best of our knowledge, this has not been discussed in other literature since.
Here, we continue the discussion for the specific case of VQAs with the goal to
parametrize the discussion with circuit characteristics (depth, width, DLA) and
discuss the bounds on VQA computation from the optimal control theorems
presented in the paper.

\subsection{Summary of Bounds on Quantum Optimal Control}
A quantum system (target of control) can be presented as a dynamical equation

\begin{equation}
    \dot{\rho} = \mathcal{L}(\rho, \controlpulse(t))~,
\end{equation}

where \(\rho\) is the density matrix representing the current state of the
system, \(\dot{\rho}\) its time evolution, \(\controlpulse\) the externally
applied control pulse, and \(\mathcal{L}\), here, the resulting Liouvillian
superoperator \cite[see][section IV]{manzano2020lindblad}. The same notation is
used for the loss function earlier, and is kept here only to be consistent with
the source. The two will not be used together in this thesis.

The dynamics are subject to the boundary condition \(\rho(t=0) = \rho_0\), and
the unitary part of \(\mathcal{L}\) must be generated by a Hamiltonian

\begin{equation}
    \hamiltonian = \hamiltonian_D + \controlpulse(t) \hamiltonian_C~,
\end{equation}

where \(\hamiltonian_D\) and \(\hamiltonian_C\) are the drift and control
Hamiltonians respectively. The dynamics can be generalized to have several
control Hamiltonians and corresponding pulses, but the extension is
straightforward and skipped here for simplicity.

Now, for choices of the control pulse \(\controlpulse\), define the set of
reachable states as the set \(\reachable\), a manifold with dimension
\(\dimD_\reachable\), which is a subset of the space of density matrices of
dimension \(\dimD_\rho\), with of course \(\dimD_\reachable \leq \dimD_\rho\).
Thus, given a goal state \(\bar{\rho}\), and an initial state \(\rho_0\) the
problem is to find a (not necessarily unique) optimal control pulse
\(\bar{\controlpulse}\) such that it drives the initial state to a final state
within an \(\epsilon\)-ball around the goal state. This can be written as a
functional minimization

\begin{equation}
    \bar{\rho}(t) = \text{arg min}_{\controlpulse(t)} 
    \mathcal{F}(\rho_0, \bar{\rho}, \controlpulse(t), [\lambda_i])~,
\end{equation}

where the functional \(\mathcal{F}\) quantifying the distance between states may
also include constraints introduced via the Lagrangian multipliers
\(\{\lambda_i\}\).

For a detailed build up to the results, see \cite{lloyd2014information}.

\subsection{Bounds on PQC Optimisation}
