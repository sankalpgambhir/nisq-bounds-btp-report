% nisq.tex

% what is nisq
\begin{frame}
    \frametitle{Noisy Intermediate-Scale Quantum Devices}

    \begin{itemize}[<+->]
        \item Near term devices
        \item Not large enough to do error checking and correction --- `noisy'
        \item Few hundred qubits expected --- `intermediate-scale'
        \item Low coherence times
    \end{itemize}

    \notes{These properties leave devices manufacturable in the forseeable future
    unable to live up to the `quantum hype', atleast in a purist's mind. }

\end{frame}

% where can it work
\begin{frame}
    \frametitle{NISQ Devices --- Utility}

    What are NISQ devices good for then?

    \pause

    \begin{itemize}
        \item Computations resilient to noise
        \item Computations with low processing times
    \end{itemize}

    \pause

    \notes{This suggests, rather than using these devices for independent computation,
    perhaps it's better to use them to run subroutines.} 
    %
    Any actual examples to speak of?

\end{frame}

\begin{frame}

    \begin{itemize}
        \item Quantum Approximate Optimization Algorithms (QAOAs)
        \item Quantum Variational Eigensolvers (QVEs)
        \item Ground State Estimation
        \item Polynomial Unconstrained Binary Optimization (PUBO)
                \footfullcite{bittel2021trainingvqa}
                \notes{H is a polynomial of the variables, which take 0/1}
    \end{itemize}


\end{frame}
